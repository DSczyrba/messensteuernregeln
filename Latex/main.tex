\input{vorlage_light}

%Für diese Version des Dokuments spezielle Anpassungen
\usepackage{array}
\newcolumntype{P}[1]{>{\centering\arraybackslash}p{#1}}
\definecolor{green}{RGB}{0, 153, 0}
\newcommand{\lat}[2]{(\textit{Lateinisch \glqq{}#1\grqq{} = #2})}
\newcommand{\merk}[1]{\glqq{}\textit{#1}\grqq{}}
\newcommand{\rot}[1]{\textcolor{red}{#1}}
\newcommand{\gruen}[1]{\textcolor{green}{#1}}
\newcommand{\schwarz}[1]{\textcolor{black}{#1}}
\newcommand{\menge}[1]{\{ #1 \}}
\newcommand{\komplement}[1]{\overline{#1}^E}
%\tikzstyle{every picture}+=[remember picture]
\everymath{\displaystyle}


\begin{document}
\frontmatter
\title{Vorbereitung Messen, Steuern \& Regeln}
\author{Luca Scheibe, Dominic Sczyrba}
\maketitle
\tableofcontents

\mainmatter
\addsec{Aufgabe 4}
\label{sec:aufgabe4}
Differenzieren und integrieren Sie!
\begin{itemize}[leftmargin=*]
    \item[a)] $f(x) = \sin x$
    \item[] $f^{\prime}(x) = \cos x$
    \item[] $\int f(x) \, dx = -\cos x$ 
    \item[b)] $f(x) = \cos x$
    \item[] $f^{\prime}(x) = -\sin x$
    \item[] $\int f(x) \, dx = \sin x$  
    \item[c)] $f(x) = \sin \left(\alpha x\right)$
    \item[] $f^{\prime}(x) = \alpha \cdot \cos \left(\alpha x\right) $
    \item[] $\int f(x) \, dx = \frac{-\cos \left(\alpha x\right)}{\alpha} + c$   
    \item[d)] $f(x) = \sin \left( \alpha x + b \right)$
    \item[] $f^{\prime}(x) =  \alpha \cdot \cos\left(\alpha x +b\right)$
    \item[] $\int f(x) \, dx = \frac{-\cos (\alpha x + b)}{\alpha}$ 
    \item[e)] $f(x) = \cos \left(\alpha x + b\right)$
    \item[] $f^{\prime}(x) = -\alpha \cdot \sin(\alpha x + b) $
    \item[] $\int f(x) \, dx = \frac{1}{\alpha} \cdot sin(\alpha x + b) = \frac{sin(\alpha x + b)}{\alpha}$ 
\end{itemize}

\addsec{Aufgabe 5}
\label{sec:aufgabe5}
\begin{itemize}[leftmargin=*]
    \item[a)] Wofür steht $\omega$?
    \item[] Der griechische Buchstabe $\omega$ steht für die Kreisfrequenz.
    \item[b)] Skizzieren Sie einen Zusammenhang $\alpha = f(t) = \omega \dot t$!
    \item[] (Skizze Lineare Funktion)  
\end{itemize}

\addsec{Aufgabe 6}
\label{sec:aufgabe6}
(Skizzen von Funktionsgraphen einfügen)

\addsec{Aufgabe 7}
\label{sec:aufgabe7}
(Skizzen von Funktionsgraphen einfügen)

\addsec{Aufgabe 8}
\label{sec:aufgabe8}
Machen Sie den Nennen rational!
\begin{align*}
    \frac{1}{\sqrt{2}} &=  & |\cdot \sqrt{2} \\
                       &= \frac{1 \cdot \sqrt{2}}{\sqrt{2} \cdot \sqrt{2}} & \\
                       &= \frac{1 \cdot \sqrt{2}}{\sqrt{4}} & \\
                       &= \underline{\underline{\frac{\sqrt{2}}{2} = \frac{1}{2} \cdot \sqrt{2}}} & \\
\end{align*}

\addsec{Aufgabe 9}
\label{sec:aufgabe9}
Schreiben Sie als Potenz um!

$\log_{b}a = c \quad \rightarrow \quad a = b^c$

\addsec{Aufgabe 10}
\label{sec:aufgabe10}
Wenden Sie die Potenzgesetze an!

\begin{itemize}[leftmargin=*]
    \item[a)] $a^x \cdot a^y = a^{(x+y)}$
    \item[b)] $\frac{a^x}{a^y} = a^{(x-y)}$
    \item[c)] $\left(a^x\right)^y = a^{(x \cdot y)}$
\end{itemize}

\addsec{Aufgabe 11}
\label{sec:aufgabe11}
Fassen Sie die Wurzeln zusammen!

\begin{itemize}[leftmargin=*]
    \item[a)] $\sqrt[x]{a} \cdot \sqrt[x]{b} = \sqrt[x]{a \cdot b }$
    \item[b)] $\frac{\sqrt[x]{a}}{\sqrt[x]{b}} = \sqrt[x]{\frac{a}{b}}$
    \item[c)] $\sqrt[x]{\sqrt[y]{a}} = \sqrt[x \cdot y]{a}$  
\end{itemize}

\addsec{Aufgabe 12}
\label{sec:aufgabe12}
Wenden Sie die Logarithmengesetze an!

\begin{itemize}[leftmargin=*]
    \item[a)] $\log_x(a \cdot b) = \log_x a + \log_x b$
    \item[b)] $\log_x\left(\frac{a}{b}\right) = \log_x a - \log_x b$
    \item[c)] $\log_x\left(a^b\right) = b \cdot \log_x a$ 
\end{itemize}

\end{document}