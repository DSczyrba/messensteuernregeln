%Alle Kapitel beginnen mit der Hauptüberschrift
\section{Einleitung}
Diese PDF wurde mit der Vorlage erstellt, um die Funktion und Formatierung dieser zu zeigen.

Die Vorlage\onlinezitat{SCZYRBA2020} ist ein Gemeinschaftsprojekt im Rahmen unseres Studiums.
Der Docker-Container\onlinezitat{HILLE2021} gehört dazu.
Die Vorlage richtet sich weitestgehend nach dem Dokument \ac{HAWA}\onlinezitat{HAWA} der \href{https://www.ba-glauchau.de/}{Staatlichen Studienakademie Glauchau}.

Weitere Hinweise befinden sich in der README.md oder im \href{https://github.com/DSczyrba/Vorlage-Latex/wiki}{Wiki}.
Eine ausführliche Dokumentation zu diesem Dokument wird folgen.

\section{Beispiele}
%! "../" ist normalerweise weder bei Pixel- noch bei Vektorgrafiken nötig. Die mitgelieferten Grafiken liegen in einem anderen Ordner
\subsection{Pixelgrafiken}
\bild[0.5]{../vorlage/ba-gc-logo}{Text zu einem Bild}{label1}
\subsection{Vektorgrafiken}
\svg[0.5]{../vorlage/ba-gc-logo}{Text zu einer \ac{SVG}-Datei}{label2}
\subsubsection{Programmcode}
\begin{code}[H]
    \begin{minted}{python}
        import asdf
        from foo import bar
        
        def yolo():
            return "glhf"
        
        class Wooo(Foo):
            def __init__(self, boo):
                self.doo = boo
                moo = 1 + 2 +3
    \end{minted}
    \caption{Beispielcode}
    \label{code:example}
\end{code}

\section{Kapitel 1}
\blindtext
