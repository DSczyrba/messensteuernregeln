%Alle Kapitel beginnen mit der Hauptüberschrift
\section{Einleitung}
Diese PDF wurde mit der Vorlage erstellt, um die Funktion und Formatierung dieser zu zeigen.

Die Vorlage\onlinezitat{SCZYRBA2020} ist ein Gemeinschaftsprojekt im Rahmen unseres Studiums.
Der Docker-Container\onlinezitat{HILLE2021} gehört dazu.
Die Vorlage richtet sich weitestgehend nach dem Dokument \ac{HAWA}\onlinezitat{HAWA} der \href{https://www.ba-glauchau.de/}{Staatlichen Studienakademie Glauchau}.

Weitere Hinweise befinden sich in der README.md oder im \href{https://github.com/DSczyrba/Vorlage-Latex/wiki}{Wiki}.
Eine ausführliche Dokumentation zu diesem Dokument wird folgen.

\section{Beispiele}
%! "../vorlage/" ist normalerweise weder bei Pixel- noch bei Vektorgrafiken nötig. Die mitgelieferten Grafiken liegen in einem anderen Ordner
\subsection{Pixelgrafiken}
Siehe Quelltext.\fn{Dort wird das Kommando \striche{bild} aufgeführt. In dieser Vorlage sind allerdings keine Bilder (im vorgesehenen Ordner) enthalten.}
%\bild[0.5]{../vorlage/ba-gc-logo}{Text zu einem Bild}{label1} %!bei Dateien in "bilder" ohne "../vorlage/"
\subsection{Vektorgrafiken}
Siehe Quelltext.
%\svg[0.5]{../vorlage/ba-gc-logo}{Text zu einer \ac{SVG}-Datei}{label2} %!bei Dateien in "bilder" ohne "../vorlage/"
\subsubsection{Programmcode}
\begin{code}[H]
    \begin{minted}{python}
        import asdf
        from foo import bar
        
        def yolo():
            return "glhf"
        
        class Wooo(Foo):
            def __init__(self, boo):
                self.doo = boo
                moo = 1 + 2 +3
    \end{minted}
    \caption{Beispielcode}
    \label{code:example}
\end{code}

\subsection{Ordnerstruktur}
    \verzeichnis{%
          .1 Vorlage-Latex.
            .2 HINWEISE.md.
            .2 LICENSE.
            .2 README.md.
            .2 sortieren.py.
            .2 Latex.
              .3 bilder.
                .4 firmenlogo.svg.
              .3 inhalt.
                .4 Abkürzungen.tex.
                .4 Anhang.tex.
                .4 Kapitel1.tex.
              .3 light.
                .4 main.tex.
                .4 README.md.
                .4 vorlage\_light.tex.
              .3 literatur.bib.
              .3 main.tex.
              .3 metadaten.sty.
              .3 vorlage.
                .4 Abstract\_Freigabeerklärung\_Bachelorthesis.tex.
                .4 ba-gc-logo.png.
                .4 ba-gc-logo.svg.
                .4 dirtree\_file.svg.
                .4 dirtree\_folder.svg.
                .4 Erklärung\_3Autoren.tex.
                .4 Erklärung\_4Autoren.tex.
                .4 Erklärung\_Praxisbeleg.tex.
                .4 Titelseite\_3Autoren.tex.
                .4 Titelseite\_4Autoren.tex.
                .4 Titelseite\_Praxisbeleg.tex.
                .4 vorlage.tex.
    }{Ein Verzeichnis-Baum}{beispielbaum}
  
\subsection{Tabellen}

\begin{table}[H]
\begin{tabularx}{\columnwidth}{|p{3cm}|X|p{.2\columnwidth}|}
\hline
Gegenstand & Beispiel & Anmerkungen \\
\hline
Kurzer Verweis auf Anhang & \litearef{sec:anhang1} & Alias: aref \\
\hline
Kurzer Verweis auf Abbildung & \litebref{beispielbaum} & Alias: bref \\
\hline
Kurzer Verweis auf Tabelle & \litetref{beispieltabelle} & Alias: tref \\
\hline
Langer Verweis auf Anhang & \fullaref{sec:anhang1} & Alias: aref \\
\hline
Langer Verweis auf Abbildung & \fullbref{beispielbaum} & Alias: bref \\
\hline
Langer Verweis auf Tabelle & \fulltref{beispieltabelle} & Alias: tref \\
\hline
\multicolumn{2}{|c|}{verbundene Spalten mit zentriertem Text} & multicolumn \\
\hline
Zeile 1 & \multirow{2}{\hsize}{verbundene Zeilen inklusive automatischer Zeilenumbrüche} & multirow \\
\cline{1-1}\cline{3-3}
Zeile 2 & & mit hsize\\
\hline
\end{tabularx}
\caption{Beispieltabelle}
\label{beispieltabelle}
\end{table}

\section{Kapitel 1}
\blindtext
