%! Anhang

\clearpage
\appendix
\clearpage

%! Section Befehl wird umgeschrieben, um Überschriften zu verbergen
%! Kann, falls Überschriften gewollt sind, entfernt oder erst später eingefügt werden.
% Beginn
\renewcommand{\section}[1]{
\par\refstepcounter{section}
\sectionmark{#1}
\addcontentsline{atoc}{section}{\bfseries\protect\numberline{\thesection}{\mdseries#1}}
\lohead{\textnormal{#1}}
}
% Ende

%! Anpassung der Darstellung von Abbildungen im Anhang
%! Eine Variante auskommentieren
%? Möglichkeit 1: ohne Nummerierung und ohne "Abbildung"
%\renewcommand{\bild}[3][1.0]{\begin{figure}[H]
%    \centering
%    \includegraphics[width=#1\columnwidth]{bilder/#2}
%    \caption*{#3}
%    \label{fig:#3}
%    \end{figure}}

%? Möglichkeit 2: mit Nummerierung und "Abbildung", aber nicht im Abbildungsverzeichnis
\renewcommand{\bild}[3][1.0]{\begin{figure}[H]
    \centering
    \includegraphics[width=#1\columnwidth]{bilder/#2}
    \caption[]{#3}
    \label{fig:#3}
    \end{figure}}

%! Anhang 1
\section{Erster Anhang}
Der erste Anhang der Arbeit.
\clearpage

%! Anhang 2
\section{Inhalt der CD}
CD mit folgenden Inhalten:
\begin{itemize}
    \item dieses Dokument
    \item \LaTeX -Dateien
    \item \href{https://www.youtube.com/watch?v=dQw4w9WgXcQ}{YouTube-Video} als Bonus
\end{itemize}
\clearpage

% Warnungen für vorgefertigte Dokumente deaktivieren
\hbadness=10000
%  Eidestattliche Erklärung und Erklärung zur Prüfung wissenschaftlicher Arbeiten
%! hier zwischen Version für einen oder mehrere Autoren umschalten
\input{inhalt/vorlage/Erklärung_Praxisbeleg}
\input{inhalt/vorlage/Erklärung_3Autoren}
\input{inhalt/vorlage/Erklärung_4Autoren}

%  Abstract und Freigabeerklärung zur Bachelorthesis
%! Falls nötig (z.B. bei Diplomarbeit) innerhalb der Datei Wortlaut anpassen, sowie Freigabeerklärung anpassen (erfordert Format Name, Vorname usw.)

%\input{inhalt/vorlage/Abstract_Freigabeerklärung_Bachelorthesis}

% Warnungen zurücksetzen
\hbadness=1000