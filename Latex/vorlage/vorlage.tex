%! LaTeX Vorlage
\documentclass[12pt, fleqn, captions=nooneline, titlepage, footsepline, headsepline, toc=sectionentrywithdots, listof=entryprefix, bibliography=totoc, parskip=half-]{scrartcl}

\usepackage{silence} %unnötige Warnungen unterdrücken
\WarningFilter{latex}{You have requested}
\WarningFilter{scrlayer-scrpage}{\headheight to low}
\WarningFilter{scrlayer-scrpage}{\footheight to low}
\WarningFilter{scrlayer-scrpage}{Very small head height detected}
\WarningFilter{fvextra}{}
\WarningFilter{lineno}{}

\pdfsuppresswarningpagegroup=1

\ProvidesPackage{metadaten}
\usepackage{metadaten}

\usepackage{tocbasic}
\usepackage[ngerman]{babel}
\usepackage[backend=biber, labeldateparts=true, style=authortitle, isbn=false, dashed=false]{biblatex}

%! Das Inhaltsverzeichnis wird an dieser Stelle formatiert.
\RedeclareSectionCommands[beforeskip=-.1\baselineskip, afterskip=.1\baselineskip, tocindent=0pt, tocnumwidth=45pt]{section,subsection,subsubsection}

%! Formatierung aller Verzeichnisse
\renewcaptionname{ngerman}{\refname}{Quellenverzeichnis}
\setuptoc{toc}{totoc}
\setuptoc{lof}{totoc}
\setuptoc{lot}{totoc}
\renewcommand*\listoflofentryname{\bfseries\figurename}
\BeforeStartingTOC[lof]{\renewcommand*\autodot{\space\space\space\space}}
\addtokomafont{captionlabel}{\bfseries}
\renewcommand*\listoflotentryname{\bfseries\tablename}
\BeforeStartingTOC[lot]{\renewcommand*\autodot{\space\space\space\space}}

%? Anhangsverzeichnis
\providecaptionname{ngerman}{\listofatocentryname}{Anhang}


\makeatletter
\AfterTOCHead[atoc]{\let\if@dynlist\if@tocleft}
\newcommand*{\useappendixtocs}{
  \renewcommand*{\ext@toc}{atoc}
  \RedeclareSectionCommands[tocindent=0pt]{section, subsection, subsubsection}
  \RedeclareSectionCommands[tocnumwidth=85pt]{section, subsection, subsubsection}
  \addtokomafont{sectionentry}{\mdseries}
  \renewcommand*\listoflofentryname{\mdseries}
  \renewcommand{\thesection}{\arabic{section}}
  \renewcommand{\@dotsep}{10000}
  }
\newcommand*{\usestandardtocs}{
  \renewcommand*{\ext@toc}{toc}
  }
\makeatother

%! Ermöglicht die Ausgabe des aktuellen Titels.
\usepackage{nameref}
\makeatletter
\newcommand*{\currentname}{\@currentlabelname}
\makeatother

%! zusätzliche LaTeX-Packages
\usepackage{amsmath}
\usepackage{amssymb}
\usepackage{amsthm}
\usepackage{tabularx}
\usepackage{multirow}
\usepackage{setspace}
\usepackage{booktabs}
\usepackage{svg}
\usepackage{graphicx}
\usepackage{float}
\usepackage[a4paper,lmargin={2.5cm},rmargin={2.5cm},tmargin={2cm},bmargin={2cm}]{geometry}
\usepackage{lineno}
\usepackage{csquotes}
\usepackage{listings}
\usepackage{tikz}
\usepackage{dirtree}
\usepackage{varwidth}

%! Centered Dirtree - https://tex.stackexchange.com/posts/100182/revisions
\makeatletter
\def\dirtree#1{%
  %%2012\let
  \DT@indent=\parindent
  \parindent=\z@
  %%2012\let
  \DT@parskip=\parskip
  \parskip=\z@
  %%2012\let
  \DT@baselineskip=\baselineskip
  \baselineskip=\DTbaselineskip
  \let\DT@strut=\strut
  \def\strut{\vrule width\z@ height0.7\baselineskip depth0.3\baselineskip}%
  \DT@counti=\z@
  \let\next\DT@readarg
  \next#1\@nil
  \dimen\z@=\hsize
  \advance\dimen\z@ -\DT@offset
  \advance\dimen\z@ -\DT@width
%  \setbox\z@=\hbox to\dimen\z@{%
  \setbox\z@=\hbox{%
%    \hsize=\dimen\z@
    \vbox{\hbox{\@nameuse{DT@body@1}}}%
  }%
  \dimen\z@=\ht\z@
  \advance\dimen0 by\dp\z@
  \advance\dimen0 by-0.7\baselineskip
  \ht\z@=0.7\baselineskip
  \dp\z@=\dimen\z@
  \par\leavevmode
  \kern\DT@offset
  \kern\DT@width
  \box\z@
  \endgraf
  \DT@countii=\@ne
  \DT@countiii=\z@
  \dimen3=\dimen\z@
  \@namedef{DT@lastlevel@1}{-0.7\baselineskip}%
  \loop
  \ifnum\DT@countii<\DT@counti
    \advance\DT@countii \@ne
    \advance\DT@countiii \@ne
    \dimen\z@=\@nameuse{DT@level@\the\DT@countii}\DT@all
    \advance\dimen\z@ by\DT@offset
    \advance\dimen\z@ by-\DT@all
    \leavevmode
    \kern\dimen\z@
    \DT@countiv=\DT@countii
    \count@=\z@
    %%2012\LOOP
    \DT@loop
      \advance\DT@countiv \m@ne
      \ifnum\@nameuse{DT@level@\the\DT@countiv} >
        \@nameuse{DT@level@\the\DT@countii}\relax
      \else
        \count@=\@ne
      \fi
    \ifnum\count@=\z@
    %%2012\REPEAT
    \DT@repeat
    \edef\DT@hsize{\the\hsize}%
    \count@=\@nameuse{DT@level@\the\DT@countii}\relax
    \dimen\z@=\count@\DT@all
    \advance\hsize by-\dimen\z@
    \setbox\z@=\vbox{\hbox{\@nameuse{DT@body@\the\DT@countii}}}%
    \hsize=\DT@hsize
    \dimen\z@=\ht\z@
    \advance\dimen\z@ by\dp\z@
    \advance\dimen\z@ by-0.7\baselineskip
    \ht\z@=0.7\baselineskip
    \dp\z@=\dimen\z@
    \@nameedef{DT@lastlevel@\the\DT@countii}{\the\dimen3}%
    \advance\dimen3 by\dimen\z@
    \advance\dimen3 by0.7\baselineskip
    \dimen\z@=\@nameuse{DT@lastlevel@\the\DT@countii}\relax
    \advance\dimen\z@ by-\@nameuse{DT@lastlevel@\the\DT@countiv}\relax
    \advance\dimen\z@ by0.3\baselineskip
    \ifnum\@nameuse{DT@level@\the\DT@countiv} <
        \@nameuse{DT@level@\the\DT@countii}\relax
      \advance\dimen\z@ by-0.5\baselineskip
    \fi
    \kern-0.5\DT@rulewidth
    \hbox{\vbox to\z@{\vss\hrule width\DT@rulewidth height\dimen\z@}}%
    \kern-0.5\DT@rulewidth
    \kern-0.5\DT@dotwidth
    \vrule width\DT@dotwidth height0.5\DT@dotwidth depth0.5\DT@dotwidth
    \kern-0.5\DT@dotwidth
    \vrule width\DT@width height0.5\DT@rulewidth depth0.5\DT@rulewidth
    \kern\DT@sep
    \hbox{\box\z@}%
    \endgraf
  \repeat
  \parindent=\DT@indent
  \parskip=\DT@parskip
  %%2012\DT@baselineskip=\baselineskip
  \baselineskip=\DT@baselineskip
  \let\strut\DT@strut
}

\makeatother


%! PageBreaks nach jeder Section
\let\oldsection\section
\renewcommand\section{\clearpage\oldsection}

%! Code-Integration im Dokument
%? Inklusive Erzeugung eines Custom-Enviroments für Programmcodes
\usepackage{minted}
%\SetupFloatingEnvironment{listing}{name=Program code}
%\SetupFloatingEnvironment{listing}{listname=List of Program Code}


\DeclareNewTOC[
  type=code,                           % Name der Umgebung
  types=codes,                         % Erweiterung (\listofschemes)
  float,                               % soll gleiten
  floatpos=H,
  tocentryentrynumberformat=\bfseries, % voreingestellte Gleitparameter
  name=Code,                           % Name in Überschriften
  listname={Programmcodeverzeichnis},  % Listenname
  % counterwithin=chapter
]{loc}
\setuptoc{loc}{totoc}

\renewcommand*{\codeformat}{%
  \codename~\thecode%
  \autodot{\space\space\space}
}
\BeforeStartingTOC[loc]{\renewcommand*\autodot{\space\space\space\space}}


%! Formelverzeichnis
\DeclareNewTOC[
  type=formel,                         % Name der Umgebung
  types=formeln,                       % Erweiterung (\listofschemes)
  float,                               % soll gleiten
  tocentryentrynumberformat=\bfseries, % voreingestellte Gleitparameter
  name=Formel,                         % Name in Überschriften
  listname={Formelverzeichnis},        % Listenname
  % counterwithin=chapter
]{lom}
\setuptoc{lom}{totoc}
\renewcommand*{\formelformat}{%
  \formelname~\theformel%
  \autodot{\space\space\space}
}
\BeforeStartingTOC[lom]{\renewcommand*\autodot{\space\space\space\space}}


%! Schriftart
%! Die HAWA schreibt Arial vor, welches in Standard LaTeX-Distributionen nicht mitgeliefert wird. Helvetica ist nahezu identisch.
\usepackage{helvet}
\usepackage{microtype}
\renewcommand{\familydefault}{\sfdefault}
%! Hyperref und PDF-meta
\usepackage[hidelinks]{hyperref}
\hypersetup{pdftitle={\titel}}
\hypersetup{pdfsubject={\kurzbeschreibung}}
\hypersetup{pdfauthor={\autoren}}
\usepackage[numbered]{bookmark}
\usepackage{acronym}
\usepackage{enumitem}

%Abkürzungsverzeichnis - Formatierung (bspw. zur korrekten Anzeige von BASH-Befehlen)
\renewcommand*{\aclabelfont}[1]{\acsfont{#1}}

%! Caption, um Tabellen und Abbildungen richtig zu beschriften
\usepackage{caption}
% Captions linksbündig, auch wenn einzeilig
\captionsetup{
  labelsep=none,
  justification=raggedright,
  singlelinecheck=false
}
\renewcommand*{\figureformat}{%
  \figurename~\thefigure%
  \autodot{\space\space\space}
}
\renewcommand*{\tableformat}{%
  \tablename~\thetable%
  \autodot{\space\space\space}
}

%! Formatierung des Literaturverzeichnis
\DeclareLabeldate{%
\field{year}
}

\DeclareExtradate{%
\scope{
\field{labelyear}
\field{year}}
}
\DeclareFieldFormat{url}{In: \url{#1}}
\DeclareFieldFormat{urldate}{\space\mkbibparens{#1}}
\DeclareFieldFormat{date}{#1\printfield{extradate}}

\urlstyle{same}
\DeclareNameFormat{author}{%
      \nameparts{#1}%
      \usebibmacro{name:family-given}
        {\MakeUppercase{\namepartfamily}}
        {\namepartgiven}
        {\namepartprefix}
        {\namepartsuffix}%
}
\DeclareFieldFormat{title}{#1}

\renewcommand{\multinamedelim}{\addsemicolon\space}
\renewcommand{\finalnamedelim}{\addsemicolon\space}
\renewcommand{\labelnamepunct}{\addcolon\space}
\renewcommand*{\finentrypunct}{}

\setlength\bibitemsep{\baselineskip}
\setlength\bibhang{0pt}

%! Formatierung der Fußnotenzitate / Literaturverzeichnis
\renewcommand*{\newunitpunct}{\addcomma\space}

%? Normales Zitat...
\DeclareCiteCommand{\zitat}[\mkbibfootnote]
  {\usebibmacro{prenote}}
  {\usebibmacro{citeindex}
   \setunit{\addnbspace}
   \bibhyperref{\printnames[author][1-3]{labelname}}
   \setunit{\labelnamepunct}
   \newunit
   \printfield{location}
   \newunit
   \printfield{labelyear}\printfield{extradate}
   \newunit
   \printfield{pages}
   }
  {\addsemicolon\space}
  {\usebibmacro{postnote}}

%? Online Zitat
\DeclareCiteCommand{\onlinezitat}[\mkbibfootnote]
  {\usebibmacro{prenote}}
  {\usebibmacro{citeindex}
   \setunit{\addnbspace}
   online:
   \bibhyperref{\printnames[author][1-3]{labelname}}
   \setunit{\labelnamepunct}
   \newunit
   \printfield{labelyear}\printfield{extradate}
   \newunit
   \printtext{(}\printfield{urlday}\printtext{.}\printfield{urlmonth}\printtext{.}\printfield{urlyear}\printtext{)}}
  {\addsemicolon\space}
  {\usebibmacro{postnote}}
\renewcommand{\bibfootnotewrapper}[1]{\bibsentence#1}

\DeclareMultiCiteCommand{\zitate}[\mkbibfootnote]{\footpartcite}{\addsemicolon\space}
\addbibresource{literatur.bib}


%! Kopf- und Fußzeile
\usepackage[automark]{scrlayer-scrpage}
\pagestyle{scrheadings}
\clearmainofpairofpagestyles
\clearplainofpairofpagestyles
\renewcommand*{\sectionmarkformat}{}
\rohead{\textnormal{\headmark}}
\lohead{}
\lofoot{}
\cofoot{}
\rofoot{\textnormal{Seite}~\pagemark}
\makeatletter
\usepackage{geometry}
\setlength{\footheight}{18.85002pt}
\geometry{a4paper,
          left=25mm,right=25mm,top=20mm,bottom=20mm,
          includehead=false,
          includefoot=false,
          headheight = \baselineskip,
          headsep = \dimexpr\Gm@tmargin-\headheight-10mm,
          footskip = \dimexpr\Gm@bmargin-10mm,
          %showframe,
          bindingoffset=0mm}
\setlength{\footheight}{\baselineskip}
\makeatother

%! Versuch von besseren Seitenumbrüchen
\clubpenalty = 10000
\widowpenalty = 10000
\displaywidowpenalty = 10000
\widowpenalties= 3 10000 10000 150

\linespread{1.3}
\newcommand\frontmatter{%
    \cleardoublepage
  \pagenumbering{Roman}}

\newcommand\mainmatter{%
    \cleardoublepage
  \pagenumbering{arabic}}

\newcommand\backmatter{%
  \if@openright
    \cleardoublepage
  \else
    \clearpage
  \fi
   }

%! Inhalts-/Anhangsverzeichnis
\DeclareNewTOC[%
  %owner=\jobname,
  tocentrystyle=tocline,
  tocentryentrynumberformat=\PrefixBy{Anhang},
  listname={Anhangverzeichnis}% Titel des Verzeichnisses
]{atoc}% Dateierweiterung (a=appendix, toc=table of contents)

\usepackage{xpatch}
\xapptocmd\appendix{%
  \useappendixtocs
  \listofatocs
  \addcontentsline{toc}{section}{Anhangverzeichnis}
  \newpage
  \pagenumbering{gobble}
  \pagestyle{scrheadings}
  \clearmainofpairofpagestyles
  \clearplainofpairofpagestyles
  \rohead{\textnormal{Anhang~\arabic{section}}}
  \lohead{\textnormal{\currentname}}
  \lofoot{}
  \cofoot{}
  \rofoot{}
}{}{}

% ! Anpassung der Minted-Umgebung, um Abstände zu vereinheitlichen und die Optik zu verbessern
\let\oldminted\minted
\let\oldendminted\endminted
\def\minted{\begingroup \vspace{-0.3cm} \oldminted}
\def\endminted{\oldendminted \vspace{-0.5cm} \endgroup}

\xapptocmd{\inputminted}{\vspace{-0.5cm}}{}{}

%Zeilennummern neu definieren, um Warnungen zu vermeiden und modern anmutende Zahlen zu nutzen
\renewcommand{\theFancyVerbLine}{\scriptsize{\arabic{FancyVerbLine}}}

%Standardformatierung für minted-Umgebung erstellen
\setminted{
    tabsize=2,
    breaklines,
    autogobble,
    fontfamily=courier,
    linenos,
    %! see https://pygments.org/demo/#try, other nice options: paraiso-light, solarized-light, rainbow_dash, gruvbox-light, stata, tango
    style=emacs
}
%Keine Zeilennnummer, wenn der einzeilige \mint-Befehl genutzt wird
\xpretocmd{\mint}{\setminted{linenos=false}}{}{}
\xpretocmd{\minted}{\setminted{linenos=true}}{}{}

%! Eigene Befehle zur erleichterten Nutzung
% Hilfsbefehle
\newcommand{\fontheightsvg}[1]{\includesvg[height=1.75ex, inkscapelatex=false]{#1}}
\newcommand{\dtfolder}{\fontheightsvg{vorlage/dirtree_folder}\hspace{0.1cm}}
\newcommand{\dtfile}{\fontheightsvg{vorlage/dirtree_file}\hspace{0.1cm}}

% Umgebungen u.Ä.
\newcommand{\fn}[1]{\footnote{\hspace{0.5em}#1}}
\newcommand{\bild}[4][1.0]{\begin{figure}[H]
  \centering
  \includegraphics[width=#1\columnwidth]{bilder/#2}
  \caption{#3}
  \label{fig:#4}
  \end{figure}}
\newcommand{\striche}[1]{\glqq #1\grqq{}}
\newcommand{\svg}[4][1.0]{\begin{figure}[H]
    \centering
    \includesvg[width=#1\columnwidth,inkscapelatex=false]{bilder/#2}
    \caption{#3}
    \label{fig:#4}
    \end{figure}}
\newcommand{\verzeichnis}[3]{\begin{figure}[H]
  % https://tex.stackexchange.com/a/99591/220899
  \renewcommand{\DTstyle}{\textrm\expandafter\raisebox{-0.7ex}}
  \centering
  \begin{varwidth}{\textwidth}
    \dirtree{#1}  
  \end{varwidth}
  \caption{#2}
  \label{fig:#3}
  \end{figure}}
\newcommand{\logisch}[1]{$``#1``$}
\newcommand{\vglink}[2]{\footnote{\hspace{0.5em}vgl.~\href{#1}{#1}~(#2)}}
\newcommand{\python}[1]{\mintinline{python}{#1}}

% Referenzierung
% Kapitel
\newcommand{\fullref}[1]{(\emph{\hyperref[{#1}]{siehe Kapitel \ref{#1} - \nameref{#1}}})}
\newcommand{\literef}[1]{\emph{\hyperref[{#1}]{Kapitel \ref{#1} - \nameref{#1}}}}
% Anhänge
\newcommand{\litearef}[1]{\emph{\hyperref[{#1}]{Anhang \ref{#1}}}}
\newcommand{\aref}[1]{\litearef{#1}} %Kompatibilität
\newcommand{\fullaref}[1]{(siehe \emph{\hyperref[{#1}]{Anhang \ref{#1}}})}
% Abbildungen
\newcommand{\litebref}[1]{\emph{\hyperref[{fig:#1}]{Abbildung \ref{fig:#1} - \nameref{fig:#1}}}}
\newcommand{\bref}[1]{\litebref{#1}} %Kompatibilität
\newcommand{\fullbref}[1]{\emph{(siehe \hyperref[{fig:#1}]{Abbildung \ref{fig:#1} - \nameref{fig:#1}}})}
% Tabellen
\newcommand{\litetref}[1]{\emph{\hyperref[{#1}]{Tabelle \ref{#1} - \nameref{#1}}}}
\newcommand{\tref}[1]{\litetref{#1}} %Einheitlichkeit
\newcommand{\fulltref}[1]{\emph{(siehe \hyperref[{#1}]{Tabelle \ref{#1} - \nameref{#1}}})}
